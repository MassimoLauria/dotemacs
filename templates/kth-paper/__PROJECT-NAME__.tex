%
% RESEARCH PAPER TEMPLATE
%=========================
%

%
% P R E A M B L E
%-----------------
%

\documentclass[11pt,twoside,a4paper,notitlepage]{article}

% Define \DONOTINSERTCOMMENTS to remove all "meta comments" from the paper
% If \DONOTINSERTCOMMENTS is *not* defined, then comments are inserted.
\def\DONOTINSERTCOMMENTS{}

%
% SOME PACKAGES
%

% Commands for simple test and loops
\usepackage{ifthen}

% Conditional compilation for conference version
\newboolean{conferenceversion}
%    \setboolean{conferenceversion}{true}
\setboolean{conferenceversion}{false}

% Fine-tuning of space
\usepackage[a4paper,margin=2.5cm]{geometry}
\usepackage{times}

% Internationalization package
\usepackage[english]{babel}

% Customized headers and footers
\usepackage{fancyhdr}

% For getting subfigures 1(a), 1(b) etc
\usepackage[sf,SF]{subfigure}

% Sam Buss's package for formatting proofs
\usepackage{bussproofs}

% For getting watermark "DRAFT" across all pages
%    \usepackage{draftwatermark}

% Instruct preamble.tex to do \usepackage{hyperref} at the right place
\newcommand{\loadhyperrefpackage}{}

%
% MACRO FILES FOR LaTeX 
%
% The general macro files live in https://svn.csc.kth.se/~jakobn/latex/
%

\input{preamble.tex}

\input{notationProofComplexity.tex}
\input{notationClauseSpacePebbling.tex}

% Local notation for this paper
%
% MACROS FOR NOTATION LOCAL TO THIS PAPER
%=========================================
%


% Local "meta" macros for this paper
%
% MACROS SPECIFIC TO "META FORMATTING" OF CURRENT PAPER
%=======================================================
%

%
% HOW TO REFER TO THE AUTHORS
%

\newcommand{\theauthorML}{the first author\xspace}
\newcommand{\TheauthorML}{The first author\xspace}

\newcommand{\theauthorMM}{the second author\xspace}
\newcommand{\TheauthorMM}{The second author\xspace}

\newcommand{\theauthorJN}{the third author\xspace}
\newcommand{\TheauthorJN}{The third author\xspace}

\newcommand{\theauthorMV}{the fourth author\xspace}
\newcommand{\TheauthorMV}{The fourth author\xspace}

%
% SWITCHING BETWEEN APPENDIX AND SECTION REFERENCES
%
% \refsecapp is defined to make it easy to change appendix references in
% the submission into section references in the full paper.
%

\ifthenelse{\boolean{conferenceversion}}
{
  \newcommand{\mysubsection}[1]{\paragraph{#1}}
  \renewcommand{\mysubsection}[1]{\textbf{\textit{#1}}}
  \renewcommand{\mysubsection}[1]{\paragraph{\textbf{#1}}}

  % Version of commands for extended abstract with appendices
  \newcommand{\refsecapp}[1]{Appendix~\ref{#1}}
  \newcommand{\Refsecapp}[1]{Appendix~\ref{#1}}
  \newcommand{\reftwosecapps}[2]{Appendices~\ref{#1} and~\ref{#2}}
  \newcommand{\Reftwosecapps}[2]{Appendices~\ref{#1} and~\ref{#2}}
  \newcommand{\Sectionappendixtext}{Appendix\xspace}
  \newcommand{\sectionappendixtext}{appendix\xspace}
  \newcommand{\Sectionsappendicestext}{Appendices\xspace}
  \newcommand{\sectionsappendicestext}{appendices\xspace}
}
{
  \newcommand{\mysubsection}[1]{\subsection{#1}}

  % Version of commands for full-length paper
 \newcommand{\refsecapp}[1]{Section~\ref{#1}}
 \newcommand{\Refsecapp}[1]{Section~\ref{#1}}
 \newcommand{\reftwosecapps}[2]{Sections~\ref{#1} and~\ref{#2}}
 \newcommand{\Reftwosecapps}[2]{Sections~\ref{#1} and~\ref{#2}}
 \newcommand{\Sectionappendixtext}{Section\xspace}
 \newcommand{\sectionappendixtext}{section\xspace}
 \newcommand{\Sectionsappendicestext}{Sections\xspace}
 \newcommand{\sectionsappendicestext}{sections\xspace}
}

%
% "META COMMENTS" TO BE TOGGLED ON OR OFF (FOR COMMENTS DURING WRITE-UP)
%

\ifthenelse{\boolean{conferenceversion}}
{}
{
\newtheoremstyle{metacommenttheoremstyle}% name
    {3pt}%      Space above
    {3pt}%      Space below
    {\sffamily \itshape \scriptsize
      % or  \footnotesize or \small
    }%         Body font
    {}%         Indent amount (empty = no indent, \parindent = para indent)
    {\bfseries \scshape \footnotesize }% Thm head font
    {:}%        Punctuation after thm head
    { }%     Space after thm head: " " = normal interword space;
    %       \newline = linebreak
    {}%         Thm head spec (can be left empty, meaning `normal')

\theoremstyle{metacommenttheoremstyle}
}
\newtheorem{jncommentcontainer}{Jakob's comment}
\newtheorem{mlcommentcontainer}{Massimo's comment}
\newtheorem{mmcommentcontainer}{Mladen's comment}
\newtheorem{mvcommentcontainer}{Marc's comment}


\ifthenelse{\isundefined{\DONOTINSERTCOMMENTS}}
{ % Turn comments ON
%      \usepackage{color}
  \usepackage[usenames,dvipsnames]{xcolor}

  \newcommand{\jncomment}[1]%
  {\begin{jncommentcontainer} \textcolor{blue}{#1} \end{jncommentcontainer}}

  \newcommand{\mlcomment}[1]%
  {\begin{mlcommentcontainer} \textcolor{OliveGreen}{#1} \end{mlcommentcontainer}}

  \newcommand{\mmcomment}[1]%
  {\begin{mmcommentcontainer} \textcolor{magenta}{#1} \end{mmcommentcontainer}}

  \newcommand{\mvcomment}[1]%
  {\begin{mvcommentcontainer} \textcolor{orange}{#1} \end{mvcommentcontainer}}

}
{ % Turn comments OFF
  \newcommand{\jncomment}[1]{}
  \newcommand{\mlcomment}[1]{}
  \newcommand{\mmcomment}[1]{}
  \newcommand{\mvcomment}[1]{}
}





% SECTIONAL NUMBERING
% secnumdepth: level of least significant sectional unit with numbered heading
% tocdepth: level of least significant sectional unit in table of contents
%    \setcounter{secnumdepth}{3}
%    \setcounter{tocdepth}{1}

% NUMBERING OF EQUATIONS
\numberwithin{equation}{section}

% DISPLAY LABELS
%
% Display the labels used in a tex file in the dvi file.  By
% default, labels are displayed not only when they are first
% named but also when they are cited (or referenced). The
% options passed disable this feature.
%
%    \providecommand*\showkeyslabelformat[1]{%
%    \fbox{\normalfont\tiny#1}%
%    %      \fbox{\normalfont\tiny\ttfamily#1}%
%    }
%    %    \usepackage[notref, notcite]{local-showkeys}

%
% D O C U M E N T
%-----------------
%

\begin{document}

%
% TITLE PAGE
%

\title{__TITLE__%
  \thanks{Any notes about the title go here.}}

\author{%
  __USER-NAME__ \\
  __ORGANISATION__\\
  \texttt{__USER-MAIL-ADDRESS__}
}

% For the date, switch to \date{\Now} to get timestamp also 
% (to keep track of most current version)
%    \date{\Now}
\date{\today}

\maketitle

%
% PAGESTYLE AND LEFT AND RIGHT RUNNING HEADS
%

% Totally empty header and footer on first page
\thispagestyle{empty}

% Only page number centered in footer (but this is issued automatically
% by article class)
%    \thispagestyle{plain}
%
% Begin new page, and start numbering pages from page 1 (i.e., reset
% page counter) with ordinary arabic numerals 1, 2, 3, ...
% (for conference submissions)
%    \newpage
%    \pagenumbering{arabic}


% \pagestyle{fancy} gives the default fancy style
\pagestyle{fancy}
% Clear all header and footer fields
\fancyhead{}
\fancyfoot{}
% We want no rule in header or footer
\renewcommand{\headrulewidth}{0pt}
\renewcommand{\footrulewidth}{0pt}

%
% Section title centered on even and subsection centered on odd pages in header
% page number centered in footer
%    \fancyhead[CE]{\slshape \leftmark}
%    \fancyhead[CO]{\slshape \rightmark}
%
% Header: title centered on even pages (entered manually), section heading
% centered on odd pages
% Footer: page number centered
\fancyhead[CE]{\slshape PAPER TITLE HERE}
\fancyhead[CO]{\slshape \nouppercase{\leftmark}}
\fancyfoot[C]{\thepage}

% Increase headheight to avoid following warning message
%    ``Package Fancyhdr Warning: \headheight is too small (12.0pt):
%    Make it at least 13.59999pt.
%    We now make it that large for the rest of the document.
%    This may cause the page layout to be inconsistent, however.''
\setlength{\headheight}{13.6pt}

%
% AND HERE THE PAPER PROPER BEGINS
%

\begin{abstract}
  \emph{Abstract goes here.}
\end{abstract}
 


\section{Introduction}
\label{sec:intro}

\emph{Introduction goes here.}

\emph{Below follow some references just to generate a bibliography at
  the end of this example file.}

Here is the standard reference on proof complexity by Cook and
Reckhow~\cite{CR79Relative}. 
%
Three good standard references on
\introduceterm{conflict-driven clause learning (CDCL)}
are~\cite{BS97UsingCSP,MS99Grasp,MMZZM01Engineering}.



\section{Some Information About Macros}
\label{sec:info}

\emph{The information below is adapted from scribing instructions, which
explains the somewhat wiseacre tone in places\ldots}

\emph{Hopefully, though, there should be some useful information here about
the available macros, which will hopefully help to produce a
consistently formatted write-up.}

\subsection{Theorem Environments, Labels, and References}
\label{sec:env-label-ref}

You can label sections by using labels
\verb+\label{sec:info}+
and
\verb+\label{sec:env-label-ref}+
and then refer to
\refsec{sec:info}
and
\refsec{sec:env-label-ref}
conveniently 
by using the macros 
\verb+\refsec{sec:info}+
and
\verb+\refsec{sec:env-label-ref}+.


\begin{theorem}[Optional name of theorem and/or reference]
  \label{th:template-theorem}
  This is a theorem.
\end{theorem}

Stating a result with a reference can be done as follows.

\begin{lemma}[\cite{Cook71ComplexityTheoremProving}]
  \label{lem:template-lemma}
  This is a lemma with a fictional citation  
  \verb+\cite{Cook71ComplexityTheoremProving}.+
\end{lemma}

A definition with both a name and a reference can be formatted like this.

\begin{definition}[Propositional proof system \cite{CR79Relative}]
  \label{def:template-def}
  Use emphasis to highlight new concepts that are being defined. For
  instance, a
  \emph{propositional proof system}
  is something that we completely fail to define here, regardless of
  the heading.
\end{definition}

\begin{proposition}
  \label{pr:template-proposition}
  This is a proposition.
\end{proposition}

\begin{proof}
  This is a proof of
  \refpr{pr:template-proposition}.
\end{proof}

Just a ``proof'' is assumed to prove the latest theorem-like
statement.  If you are proving some previous statement, you can
specify which in the following way.

\begin{proof}[Proof of \reflem{lem:template-lemma}]
  This is a proof of some previously stated result, e.g.,
  \reflem{lem:template-lemma}. 

  Note that a default QED box is added at the end of each proof.
  If you do not get it where you want, e.g., if the proof ends with a
  list,  you can force placement of the box by issuing the command
  \verb+\qedhere+ (but this is not needed here).
\end{proof}

\begin{observation}
  \label{obs:template-observation}
  This is an observation.
\end{observation}

\begin{proof}[Proof sketch]
  This is not a full proof, but a proof sketch.
\end{proof}

You can use
\verb+\refth{th:template-theorem}+
to refer to
\refth{th:template-theorem}.

You can use
\verb+\reflem{lem:template-lemma}+
to refer to
\reflem{lem:template-lemma}.

You can use
\verb+\refpr{pr:template-proposition}+
to refer to
\refpr{pr:template-proposition}.

You can use
\verb+\refobs{obs:template-observation}+
to refer to
\refobs{obs:template-observation}.

You can use
\verb+\refdef{def:template-def}+
to refer to
\refdef{def:template-def}.

\subsection{General Math Editing Info}

Following principles in the AMS math packages,
many typesetting commands with delimiters are available in three
versions:
\verb+\command+ with small delimiters,
\verb+\Command+ with somewhat larger ones, and
\verb+\COMMAND+ with very large delimiters (use only in displayed
math).

For instance:
\begin{itemize}
\item 
\verb+$\bigoh{n}$+ 
yields
$\bigoh{n}$
and
\verb+$\Bigoh{n^k}$+
yields
$\Bigoh{n^k}$
with slightly larger parantheses.
\item 
\verb+$\bigomega{\log n}$+ 
yields
$\bigomega{\log n}$
and
\verb+$\Bigomega{\frac{1}{n}}$+
yields
$\Bigomega{\frac{1}{n}}$.

\item
  \verb+$\bigtheta{n}$+ becomes
  $\bigtheta{n}$
  while
  \verb+$\Bigtheta{\frac{n}{\log n}}$+
  turns into
  $\Bigtheta{\frac{n}{\log n}}$.
  
\item
  \verb+$\abs{x}$+ 
  becomes
  $\abs{x}$
  and
  \verb+$\Abs{x^y}$+
  becomes
  $\Abs{x^y}$.
\end{itemize}

Other commands that work the same way (i.e., have
versions with delimiters of adaptable sizes) are:
\begin{enumerate}
\item 
  \verb+\set+ denoting a set:
  $\set{1,2,3,\ldots}$.
\item 
  \verb+\setdescr+
  denoting a set described by some parameter
  (use
  \verb+\Setdescr+
  for this example 
  to get a nicer look):
  $\Setdescr{2^i}{i \in \Nplus}$.
\item 
  \verb+\setsize+
  denoting the size of a set:
  $\setsize{\vertices{G}} = n$.
\item 
  \verb+\maxofset+ 
  and
  \verb+\minofset+ 
  taking the maximum and minimum over a set as in, for instance, 
  $\minofset[\mid]{\clspaceof{\pi}}{\derivof{\pi}{F}{\emptycl}}$.
\item 
  \verb+\maxofexpr+
  and
  \verb+\minofexpr+
  taking the maximum/minimum over an ``expression'' as in, for instance,
  $\maxofexpr[D \in \pi]{\widthofarg{D}}$.
\end{enumerate}

Here is the preferred way to display mathematics without a getting a
numbered equation to reference
using
\verb+\begin{equation*}+:
\begin{equation*}
  x^2 - x = 0
\end{equation*}
and
here is the preferred way to display mathematics with a number
using
\verb+\begin{equation}+:
\begin{equation}
  \label{eq:example-equation}
  x + \olnot{x} - 1 = 0
\end{equation}
You can use
\verb+\refeq{eq:example-equation}+
to refer to this equation, which will give a reference that looks like
\refeq{eq:example-equation}.

\subsection{Proof Complexity Notation}


Sometimes we want to specify in a subscript the proof system in question when
discussing, e.g., proof complexity measures. In such a case one can write
\verb+\resnot+
to denote general resolution
$\resnot$,
\verb+\treeresnot+
to denote tree-like resolution
$\treeresnot$,
\verb+\cpnot+
to denote cutting planes
$\cpnot$,
\verb+\pcnot+
to denote polynomial calculus
$\pcnot$,
and
\verb+\pcrnot+
to denote polynomial calculus resolution
$\pcrnot$.
These macros do not look very pretty in running text, however---here
CP, PC and PCR seem better.

For monomial space in PC and PCR, we can code, for instance:
\begin{itemize}
\item 
$\mspaceof{\pi}$
(\verb+\mspaceof{\pi}+).

\item
$\mspaceref[\pcnot]{F}$
(\verb+\mspaceref[\pcnot]{F}+).

\item
$\Mspaceref[\pcrnot]{\ephpnot{m}{n}}$
(\verb+\Mspaceref[\pcrnot]{\ephpnot{m}{n}}+).

\item
$\Mspacederiv[\pcrnot]{\ephpnot{m}{n}}{\pconf}$
(\verb+\Mspacederiv[\pcrnot]{\ephpnot{m}{n}}{\pconf}+).

\end{itemize}



For degree, we can write
\begin{itemize}
\item 
$\mdegreeof{\pi}$
(\verb+\mdegreeof{\pi}+).

\item
$\mdegreeref[\pcnot]{F}$
(\verb+\mdegreeref[\pcnot]{F}+).

\item
$\Mdegreeref[\pcrnot]{\ephpnot{m}{n}}$
(\verb+\Mdegreeref[\pcrnot]{\ephpnot{m}{n}}+).

\item
$\Mdegreederiv[\pcrnot]{\ephpnot{m}{n}}{\pconf}$
(\verb+\Mdegreederiv[\pcrnot]{\ephpnot{m}{n}}{\pconf}+).
\end{itemize}

For size, we can write
\begin{itemize}
\item 
$\sizeofarg{\pi}$
(\verb+\sizeofarg{\pi}+).

\item
$\sizeref[\pcnot]{F}$
(\verb+\sizeref[\pcnot]{F}+).

\item
$\Sizeref[\pcrnot]{\ephpnot{m}{n}}$
(\verb+\Sizeref[\pcrnot]{\ephpnot{m}{n}}+).

\item
$\Sizederiv[\pcrnot]{\ephpnot{m}{n}}{\pconf}$
(\verb+\Sizederiv[\pcrnot]{\ephpnot{m}{n}}{\pconf}+).

\end{itemize}



To denote the length of refuting a CNF formula $F$ in general
resolution we can use the macro
\verb+$\lengthref[\resnot]{F}$+
which looks like
$\lengthref[\resnot]{F}$.
The length of refuting a CNF formula $F$ in tree-like resolution is denoted
by the macro
\verb+$\lengthref[\treeresnot]{F}$+
which looks like
$\lengthref[\treeresnot]{F}$.

The width of refuting $\fstd$ (in general resolution, although for
this particular measure it does not matter) is written
\verb+$\widthref[\resnot]{F}$+
and looks like
$\widthref[\resnot]{F}$,
the clause space is written
\verb+$\clspaceref[\resnot]{F}$+
and looks like
$\clspaceref[\resnot]{F}$,
and for the total space we can write
\verb+$\totspaceref[\resnot]{F}$+
which looks like
$\totspaceref[\resnot]{F}$.

When the proof system under discussion is perfectly clear from
context, however, we might skip the optional argument in brackets and
just write
\verb+$\widthref{F}$+
for
$\widthref{F}$
instead.


When measuring a concrete proof $\pi$, the proof system is
presumably fixed and so it does not make too much sense to specify the
proof system one extra time. But the same optional argument in
brackets is supported if we want to do so.
Thus, we can use the macros
\verb+$\lengthofarg{\pi}$+,
i.e., 
$\lengthofarg{\pi}$,
and
\verb+$\widthofarg{\pi}$+,
i.e., $\widthofarg{\pi}$,
with the extra suffix
\verb+arg+
appended
because of unfortunate naming collisions with standard \LaTeX{} packages, and
\verb+$\clspaceof{\pi}$+,
i.e., $\clspaceof{\pi}$,
\verb+$\totspaceof{\pi}$+,
i.e., $\totspaceof{\pi}$.
%    ,
%    but we can also write, for instance
%    \verb+$\clspaceof[\resnot]{\pi}$+
%    to get
%    $\clspaceof[\resnot]{\pi}$
%    or
%    \verb+$\totspaceof[\resnot]{\pi}$+
%    to get
%    $\totspaceof[\resnot]{\pi}$.
%    

\subsection{Some Symbols}

Here is a list of some symbols (please tell me afterwards if there was something you
really missed here and I will add it):
\begin{itemize}
\item 
  $\sum_{i=1}^{n}$ coded as \verb+$\sum_{i=1}^{n}$+
\item 
  $\prod_{i=1}^{n}$ coded as \verb+$\prod_{i=1}^{n}$+
\item 
  $\to \infty$ coded as \verb+$\to \infty$+
\item 
  $\impl$ coded as \verb+$\impl$+
  and
  $\nimpl$ coded as \verb+$\nimpl$+
\end{itemize}

\subsection{Inference Rules and Derivations}

Please typeset any inference rules and derivations as
follows below, using the
\LaTeX{} tools by Sam Buss.

Just one resolution inference can be done like this:
\begin{equation}
  \AxiomC{$C_1$}
  \AxiomC{$C_2$}
  \BinaryInfC{$C_3$}
  \DisplayProof  
\end{equation}

A couple of resolution inferences ``in parallel'' can be coded like
this:
\begin{gather}
  \AxiomC{$C_1$}
  \AxiomC{$C_2$}
  \BinaryInfC{$C_3$}
  \DisplayProof  
  \\
  \AxiomC{$D_1$}
  \AxiomC{$D_2$}
  \BinaryInfC{$D_3$}
  \DisplayProof  
  \\
  \AxiomC{$E_1$}
  \AxiomC{$E_2$}
  \BinaryInfC{$E_3$}
  \DisplayProof  
\end{gather}
    
And finally, 
to code a derivation with more structure, do like this:
\begin{equation}
  \AxiomC{$C_1$}
  \AxiomC{$C_2$}
  \BinaryInfC{$C_3$}
  \AxiomC{$C_4$}
  \BinaryInfC{$C_5$}
  \AxiomC{$C_6$}
  \BinaryInfC{$C_7$}
  \noLine
  \UnaryInfC{$\vdots$}
  \noLine
  \UnaryInfC{$C_{n-2}$}
  \AxiomC{$C_{n-1}$}
  \BinaryInfC{$C_n$}
  \DisplayProof  
\end{equation}




\section{Concluding Remarks}
\label{sec:conclusion}

\emph{Concluding remarks go here.}


\section*{Acknowledgements}

\emph{Acknowledgements go here.}


Standard funding blurb:


The authors were funded by the
European Research Council under the European Union's Seventh Framework
Programme \mbox{(FP7/2007--2013) /} ERC grant agreement no.~279611.
\TheauthorJN 
was also supported by
Swedish Research Council grants 
\mbox{621-2010-4797}
and
\mbox{621-2012-5645}.



%
% BIBLIOGRAPHY
%

\bibliography{refArticles,refBooks,refOther}

%    \bibliographystyle{plain}   % standard BibTeX (numbers as labels)
%    \bibliographystyle{abbrv}   % Like standard, but more compact entries
\bibliographystyle{alpha}

\end{document}
