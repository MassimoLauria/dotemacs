
\section{Some Information About Macros}
\label{sec:info}

\emph{The information below is adapted from scribing instructions, which
explains the somewhat wiseacre tone in places\ldots}

\emph{Hopefully, though, there should be some useful information here about
the available macros, which will hopefully help to produce a
consistently formatted write-up.}

\subsection{Theorem Environments, Labels, and References}
\label{sec:env-label-ref}

You can label sections by using labels
\verb+\label{sec:info}+
and
\verb+\label{sec:env-label-ref}+
and then refer to
\refsec{sec:info}
and
\refsec{sec:env-label-ref}
conveniently 
by using the macros 
\verb+\refsec{sec:info}+
and
\verb+\refsec{sec:env-label-ref}+.


\begin{theorem}[Optional name of theorem and/or reference]
  \label{th:template-theorem}
  This is a theorem.
\end{theorem}

Stating a result with a reference can be done as follows.

\begin{lemma}[\cite{Cook71ComplexityTheoremProving}]
  \label{lem:template-lemma}
  This is a lemma with a fictional citation  
  \verb+\cite{Cook71ComplexityTheoremProving}.+
\end{lemma}

A definition with both a name and a reference can be formatted like this.

\begin{definition}[Propositional proof system \cite{CR79Relative}]
  \label{def:template-def}
  Use emphasis to highlight new concepts that are being defined. For
  instance, a
  \emph{propositional proof system}
  is something that we completely fail to define here, regardless of
  the heading.
\end{definition}

\begin{proposition}
  \label{pr:template-proposition}
  This is a proposition.
\end{proposition}

\begin{proof}
  This is a proof of
  \refpr{pr:template-proposition}.
\end{proof}

Just a ``proof'' is assumed to prove the latest theorem-like
statement.  If you are proving some previous statement, you can
specify which in the following way.

\begin{proof}[Proof of \reflem{lem:template-lemma}]
  This is a proof of some previously stated result, e.g.,
  \reflem{lem:template-lemma}. 

  Note that a default QED box is added at the end of each proof.
  If you do not get it where you want, e.g., if the proof ends with a
  list,  you can force placement of the box by issuing the command
  \verb+\qedhere+ (but this is not needed here).
\end{proof}

\begin{observation}
  \label{obs:template-observation}
  This is an observation.
\end{observation}

\begin{proof}[Proof sketch]
  This is not a full proof, but a proof sketch.
\end{proof}

You can use
\verb+\refth{th:template-theorem}+
to refer to
\refth{th:template-theorem}.

You can use
\verb+\reflem{lem:template-lemma}+
to refer to
\reflem{lem:template-lemma}.

You can use
\verb+\refpr{pr:template-proposition}+
to refer to
\refpr{pr:template-proposition}.

You can use
\verb+\refobs{obs:template-observation}+
to refer to
\refobs{obs:template-observation}.

You can use
\verb+\refdef{def:template-def}+
to refer to
\refdef{def:template-def}.

\subsection{General Math Editing Info}

Following principles in the AMS math packages,
many typesetting commands with delimiters are available in three
versions:
\verb+\command+ with small delimiters,
\verb+\Command+ with somewhat larger ones, and
\verb+\COMMAND+ with very large delimiters (use only in displayed
math).

For instance:
\begin{itemize}
\item 
\verb+$\bigoh{n}$+ 
yields
$\bigoh{n}$
and
\verb+$\Bigoh{n^k}$+
yields
$\Bigoh{n^k}$
with slightly larger parantheses.
\item 
\verb+$\bigomega{\log n}$+ 
yields
$\bigomega{\log n}$
and
\verb+$\Bigomega{\frac{1}{n}}$+
yields
$\Bigomega{\frac{1}{n}}$.

\item
  \verb+$\bigtheta{n}$+ becomes
  $\bigtheta{n}$
  while
  \verb+$\Bigtheta{\frac{n}{\log n}}$+
  turns into
  $\Bigtheta{\frac{n}{\log n}}$.
  
\item
  \verb+$\abs{x}$+ 
  becomes
  $\abs{x}$
  and
  \verb+$\Abs{x^y}$+
  becomes
  $\Abs{x^y}$.
\end{itemize}

Other commands that work the same way (i.e., have
versions with delimiters of adaptable sizes) are:
\begin{enumerate}
\item 
  \verb+\set+ denoting a set:
  $\set{1,2,3,\ldots}$.
\item 
  \verb+\setdescr+
  denoting a set described by some parameter
  (use
  \verb+\Setdescr+
  for this example 
  to get a nicer look):
  $\Setdescr{2^i}{i \in \Nplus}$.
\item 
  \verb+\setsize+
  denoting the size of a set:
  $\setsize{\vertices{G}} = n$.
\item 
  \verb+\maxofset+ 
  and
  \verb+\minofset+ 
  taking the maximum and minimum over a set as in, for instance, 
  $\minofset[\mid]{\clspaceof{\pi}}{\derivof{\pi}{F}{\emptycl}}$.
\item 
  \verb+\maxofexpr+
  and
  \verb+\minofexpr+
  taking the maximum/minimum over an ``expression'' as in, for instance,
  $\maxofexpr[D \in \pi]{\widthofarg{D}}$.
\end{enumerate}

Here is the preferred way to display mathematics without a getting a
numbered equation to reference
using
\verb+\begin{equation*}+:
\begin{equation*}
  x^2 - x = 0
\end{equation*}
and
here is the preferred way to display mathematics with a number
using
\verb+\begin{equation}+:
\begin{equation}
  \label{eq:example-equation}
  x + \olnot{x} - 1 = 0
\end{equation}
You can use
\verb+\refeq{eq:example-equation}+
to refer to this equation, which will give a reference that looks like
\refeq{eq:example-equation}.

\subsection{Proof Complexity Notation}


Sometimes we want to specify in a subscript the proof system in question when
discussing, e.g., proof complexity measures. In such a case one can write
\verb+\resnot+
to denote general resolution
$\resnot$,
\verb+\treeresnot+
to denote tree-like resolution
$\treeresnot$,
\verb+\cpnot+
to denote cutting planes
$\cpnot$,
\verb+\pcnot+
to denote polynomial calculus
$\pcnot$,
and
\verb+\pcrnot+
to denote polynomial calculus resolution
$\pcrnot$.
These macros do not look very pretty in running text, however---here
CP, PC and PCR seem better.

For monomial space in PC and PCR, we can code, for instance:
\begin{itemize}
\item 
$\mspaceof{\pi}$
(\verb+\mspaceof{\pi}+).

\item
$\mspaceref[\pcnot]{F}$
(\verb+\mspaceref[\pcnot]{F}+).

\item
$\Mspaceref[\pcrnot]{\ephpnot{m}{n}}$
(\verb+\Mspaceref[\pcrnot]{\ephpnot{m}{n}}+).

\item
$\Mspacederiv[\pcrnot]{\ephpnot{m}{n}}{\pconf}$
(\verb+\Mspacederiv[\pcrnot]{\ephpnot{m}{n}}{\pconf}+).

\end{itemize}



For degree, we can write
\begin{itemize}
\item 
$\mdegreeof{\pi}$
(\verb+\mdegreeof{\pi}+).

\item
$\mdegreeref[\pcnot]{F}$
(\verb+\mdegreeref[\pcnot]{F}+).

\item
$\Mdegreeref[\pcrnot]{\ephpnot{m}{n}}$
(\verb+\Mdegreeref[\pcrnot]{\ephpnot{m}{n}}+).

\item
$\Mdegreederiv[\pcrnot]{\ephpnot{m}{n}}{\pconf}$
(\verb+\Mdegreederiv[\pcrnot]{\ephpnot{m}{n}}{\pconf}+).
\end{itemize}

For size, we can write
\begin{itemize}
\item 
$\sizeofarg{\pi}$
(\verb+\sizeofarg{\pi}+).

\item
$\sizeref[\pcnot]{F}$
(\verb+\sizeref[\pcnot]{F}+).

\item
$\Sizeref[\pcrnot]{\ephpnot{m}{n}}$
(\verb+\Sizeref[\pcrnot]{\ephpnot{m}{n}}+).

\item
$\Sizederiv[\pcrnot]{\ephpnot{m}{n}}{\pconf}$
(\verb+\Sizederiv[\pcrnot]{\ephpnot{m}{n}}{\pconf}+).

\end{itemize}



To denote the length of refuting a CNF formula $F$ in general
resolution we can use the macro
\verb+$\lengthref[\resnot]{F}$+
which looks like
$\lengthref[\resnot]{F}$.
The length of refuting a CNF formula $F$ in tree-like resolution is denoted
by the macro
\verb+$\lengthref[\treeresnot]{F}$+
which looks like
$\lengthref[\treeresnot]{F}$.

The width of refuting $\fstd$ (in general resolution, although for
this particular measure it does not matter) is written
\verb+$\widthref[\resnot]{F}$+
and looks like
$\widthref[\resnot]{F}$,
the clause space is written
\verb+$\clspaceref[\resnot]{F}$+
and looks like
$\clspaceref[\resnot]{F}$,
and for the total space we can write
\verb+$\totspaceref[\resnot]{F}$+
which looks like
$\totspaceref[\resnot]{F}$.

When the proof system under discussion is perfectly clear from
context, however, we might skip the optional argument in brackets and
just write
\verb+$\widthref{F}$+
for
$\widthref{F}$
instead.


When measuring a concrete proof $\pi$, the proof system is
presumably fixed and so it does not make too much sense to specify the
proof system one extra time. But the same optional argument in
brackets is supported if we want to do so.
Thus, we can use the macros
\verb+$\lengthofarg{\pi}$+,
i.e., 
$\lengthofarg{\pi}$,
and
\verb+$\widthofarg{\pi}$+,
i.e., $\widthofarg{\pi}$,
with the extra suffix
\verb+arg+
appended
because of unfortunate naming collisions with standard \LaTeX{} packages, and
\verb+$\clspaceof{\pi}$+,
i.e., $\clspaceof{\pi}$,
\verb+$\totspaceof{\pi}$+,
i.e., $\totspaceof{\pi}$.
%    ,
%    but we can also write, for instance
%    \verb+$\clspaceof[\resnot]{\pi}$+
%    to get
%    $\clspaceof[\resnot]{\pi}$
%    or
%    \verb+$\totspaceof[\resnot]{\pi}$+
%    to get
%    $\totspaceof[\resnot]{\pi}$.
%    

\subsection{Some Symbols}

Here is a list of some symbols (please tell me afterwards if there was something you
really missed here and I will add it):
\begin{itemize}
\item 
  $\sum_{i=1}^{n}$ coded as \verb+$\sum_{i=1}^{n}$+
\item 
  $\prod_{i=1}^{n}$ coded as \verb+$\prod_{i=1}^{n}$+
\item 
  $\to \infty$ coded as \verb+$\to \infty$+
\item 
  $\impl$ coded as \verb+$\impl$+
  and
  $\nimpl$ coded as \verb+$\nimpl$+
\end{itemize}

\subsection{Inference Rules and Derivations}

Please typeset any inference rules and derivations as
follows below, using the
\LaTeX{} tools by Sam Buss.

Just one resolution inference can be done like this:
\begin{equation}
  \AxiomC{$C_1$}
  \AxiomC{$C_2$}
  \BinaryInfC{$C_3$}
  \DisplayProof  
\end{equation}

A couple of resolution inferences ``in parallel'' can be coded like
this:
\begin{gather}
  \AxiomC{$C_1$}
  \AxiomC{$C_2$}
  \BinaryInfC{$C_3$}
  \DisplayProof  
  \\
  \AxiomC{$D_1$}
  \AxiomC{$D_2$}
  \BinaryInfC{$D_3$}
  \DisplayProof  
  \\
  \AxiomC{$E_1$}
  \AxiomC{$E_2$}
  \BinaryInfC{$E_3$}
  \DisplayProof  
\end{gather}
    
And finally, 
to code a derivation with more structure, do like this:
\begin{equation}
  \AxiomC{$C_1$}
  \AxiomC{$C_2$}
  \BinaryInfC{$C_3$}
  \AxiomC{$C_4$}
  \BinaryInfC{$C_5$}
  \AxiomC{$C_6$}
  \BinaryInfC{$C_7$}
  \noLine
  \UnaryInfC{$\vdots$}
  \noLine
  \UnaryInfC{$C_{n-2}$}
  \AxiomC{$C_{n-1}$}
  \BinaryInfC{$C_n$}
  \DisplayProof  
\end{equation}


